\documentclass{article}\usepackage[]{graphicx}\usepackage[]{xcolor}
% maxwidth is the original width if it is less than linewidth
% otherwise use linewidth (to make sure the graphics do not exceed the margin)
\makeatletter
\def\maxwidth{ %
  \ifdim\Gin@nat@width>\linewidth
    \linewidth
  \else
    \Gin@nat@width
  \fi
}
\makeatother

\definecolor{fgcolor}{rgb}{0.345, 0.345, 0.345}
\newcommand{\hlnum}[1]{\textcolor[rgb]{0.686,0.059,0.569}{#1}}%
\newcommand{\hlsng}[1]{\textcolor[rgb]{0.192,0.494,0.8}{#1}}%
\newcommand{\hlcom}[1]{\textcolor[rgb]{0.678,0.584,0.686}{\textit{#1}}}%
\newcommand{\hlopt}[1]{\textcolor[rgb]{0,0,0}{#1}}%
\newcommand{\hldef}[1]{\textcolor[rgb]{0.345,0.345,0.345}{#1}}%
\newcommand{\hlkwa}[1]{\textcolor[rgb]{0.161,0.373,0.58}{\textbf{#1}}}%
\newcommand{\hlkwb}[1]{\textcolor[rgb]{0.69,0.353,0.396}{#1}}%
\newcommand{\hlkwc}[1]{\textcolor[rgb]{0.333,0.667,0.333}{#1}}%
\newcommand{\hlkwd}[1]{\textcolor[rgb]{0.737,0.353,0.396}{\textbf{#1}}}%
\let\hlipl\hlkwb

\usepackage{framed}
\makeatletter
\newenvironment{kframe}{%
 \def\at@end@of@kframe{}%
 \ifinner\ifhmode%
  \def\at@end@of@kframe{\end{minipage}}%
  \begin{minipage}{\columnwidth}%
 \fi\fi%
 \def\FrameCommand##1{\hskip\@totalleftmargin \hskip-\fboxsep
 \colorbox{shadecolor}{##1}\hskip-\fboxsep
     % There is no \\@totalrightmargin, so:
     \hskip-\linewidth \hskip-\@totalleftmargin \hskip\columnwidth}%
 \MakeFramed {\advance\hsize-\width
   \@totalleftmargin\z@ \linewidth\hsize
   \@setminipage}}%
 {\par\unskip\endMakeFramed%
 \at@end@of@kframe}
\makeatother

\definecolor{shadecolor}{rgb}{.97, .97, .97}
\definecolor{messagecolor}{rgb}{0, 0, 0}
\definecolor{warningcolor}{rgb}{1, 0, 1}
\definecolor{errorcolor}{rgb}{1, 0, 0}
\newenvironment{knitrout}{}{} % an empty environment to be redefined in TeX

\usepackage{alltt}
\usepackage[margin=1.0in]{geometry} % To set margins
\usepackage{amsmath}  % This allows me to use the align functionality.
                      % If you find yourself trying to replicate
                      % something you found online, ensure you're
                      % loading the necessary packages!
\usepackage{amsfonts} % Math font
\usepackage{fancyvrb}
\usepackage{hyperref} % For including hyperlinks
\usepackage[shortlabels]{enumitem}% For enumerated lists with labels specified
                                  % We had to run tlmgr_install("enumitem") in R
\usepackage{float}    % For telling R where to put a table/figure
\usepackage{natbib}        %For the bibliography
\bibliographystyle{apalike}%For the bibliography
\IfFileExists{upquote.sty}{\usepackage{upquote}}{}
\begin{document}

\begin{enumerate}
%%%%%%%%%%%%%%%%%%%%%%%%%%%%%%%%%%%%%%%%%%%%%%%%%%%%%%%%%%%%%%%%%%%%%%%%%%%%%%%%
%%%%%%%%%%%%%%%%%%%%%%%%%%%%%%%%%%%%%%%%%%%%%%%%%%%%%%%%%%%%%%%%%%%%%%%%%%%%%%%%
% QUESTION 1
%%%%%%%%%%%%%%%%%%%%%%%%%%%%%%%%%%%%%%%%%%%%%%%%%%%%%%%%%%%%%%%%%%%%%%%%%%%%%%%%
%%%%%%%%%%%%%%%%%%%%%%%%%%%%%%%%%%%%%%%%%%%%%%%%%%%%%%%%%%%%%%%%%%%%%%%%%%%%%%%%
\item This week's Problem of the Week in Math is described as follows:
\begin{quotation}
  \textit{There are thirty positive integers less than 100 that share a certain 
  property. Your friend, Blake, wrote them down in the table to the left. But 
  Blake made a mistake! One of the numbers listed is wrong and should be replaced 
  with another. Which number is incorrect, what should it be replaced with, and 
  why?}
\end{quotation}
The numbers are listed below.
\begin{center}
  \begin{tabular}{ccccc}
    6 & 10 & 14 & 15 & 21\\
    22 & 26 & 33 & 34 & 35\\
    38 & 39 & 46 & 51 & 55\\
    57 & 58 & 62 & 65 & 69\\
    75 & 77 & 82 & 85 & 86\\
    87 & 91 & 93 & 94 & 95
  \end{tabular}
\end{center}
Use the fact that the ``certain'' property is that these numbers are all supposed
to be the product of \emph{unique} prime numbers to find and fix the mistake that
Blake made.\\
\textbf{Reminder:} Code your solution in an \texttt{R} script and copy it over
to this \texttt{.Rnw} file.\\
\textbf{Hint:} You may find the \verb|%in%| operator and the \verb|setdiff()| function to be helpful.\\

\textbf{Solution:} 
% Write your answer and explanations here.

\begin{knitrout}\scriptsize
\definecolor{shadecolor}{rgb}{0.969, 0.969, 0.969}\color{fgcolor}\begin{kframe}
\begin{alltt}
\hlcom{# install the package and load the library for numbers}
\hlcom{# provides number-theoretic functions for factorization, prime numbers, etc.}
\hlkwd{library}\hldef{(numbers)}

\hlcom{# define the numbers given from Blake}
\hldef{blake_numbers} \hlkwb{<-} \hlkwd{c}\hldef{(}\hlnum{6}\hldef{,} \hlnum{10}\hldef{,} \hlnum{14}\hldef{,} \hlnum{15}\hldef{,} \hlnum{21}\hldef{,} \hlnum{22}\hldef{,} \hlnum{26}\hldef{,} \hlnum{33}\hldef{,} \hlnum{34}\hldef{,} \hlnum{35}\hldef{,} \hlnum{38}\hldef{,} \hlnum{39}\hldef{,} \hlnum{46}\hldef{,} \hlnum{51}\hldef{,} \hlnum{55}\hldef{,}
                   \hlnum{57}\hldef{,} \hlnum{58}\hldef{,} \hlnum{62}\hldef{,} \hlnum{65}\hldef{,} \hlnum{69}\hldef{,} \hlnum{75}\hldef{,} \hlnum{77}\hldef{,} \hlnum{82}\hldef{,} \hlnum{85}\hldef{,} \hlnum{86}\hldef{,} \hlnum{87}\hldef{,} \hlnum{91}\hldef{,} \hlnum{93}\hldef{,} \hlnum{94}\hldef{,} \hlnum{95}\hldef{)}

\hlcom{# check if the number is a product of unique prime numbers}
\hlcom{# return true if }
\hldef{is_product_of_unique_primes} \hlkwb{<-} \hlkwa{function}\hldef{(}\hlkwc{n}\hldef{) \{}
  \hldef{factors} \hlkwb{<-} \hlkwd{primeFactors}\hldef{(n)}
  \hlkwd{return}\hldef{(}\hlkwd{length}\hldef{(factors)} \hlopt{==} \hlnum{2} \hlopt{&&} \hlkwd{length}\hldef{(factors)} \hlopt{==} \hlkwd{length}\hldef{(}\hlkwd{unique}\hldef{(factors)))}
\hldef{\}}

\hlcom{# find valid unique prime products less than 100 (numbers 1-99)}
\hlcom{# sapply(1:99, is_product_of_unique_primes) returns a logical vector}
\hlcom{# which extracts the indicies where the function returns TRUE (valid #'s)}
\hlcom{# the result is a vector of numbers that are products of unique prime factors}

\hldef{valid_numbers} \hlkwb{<-} \hlkwd{which}\hldef{(}\hlkwd{sapply}\hldef{(}\hlnum{1}\hlopt{:}\hlnum{99}\hldef{, is_product_of_unique_primes))}

\hlcom{# identify the incorrect number}
\hldef{(incorrect_number} \hlkwb{<-} \hlkwd{setdiff}\hldef{(blake_numbers,valid_numbers))}
\end{alltt}
\begin{verbatim}
## [1] 75
\end{verbatim}
\begin{alltt}
\hlcom{# identify missing numbers}
\hldef{(missing_number} \hlkwb{<-} \hlkwd{setdiff}\hldef{(valid_numbers, blake_numbers))}
\end{alltt}
\begin{verbatim}
## [1] 74
\end{verbatim}
\end{kframe}
\end{knitrout}
\end{enumerate}

Blake's list of numbers is supposed to contain only numbers that are products of
unique prime factors (i.e., products of two distinct prime numbers). However, 
Blake made a mistake by including an incorrect number in the list. Our goal is 
to identify which number is incorrect and find the correct number to replace it. 
To do this, we implemented four key steps. First, we defined a function to check
if a number is a product of unique primes. We used the \texttt{primeFactors(n)} 
function from the \texttt{numbers} package \citep{numbers} to find the prime 
factors of a given number. The function \texttt{is\_product\_of\_unique\_primes}
ensures that the number has exactly two prime factors and that the factors are
distinct. Next, we generated a list of valid numbers that meet the criteria. 
Using \texttt{sapply(1:99, is\_product\_of\_unique\_primes)}, we applied the function 
to numbers 1 to 99 to see which satisfy our conditions. The \texttt{which()} 
function extracted the numbers where the function returned \texttt{TRUE}, 
giving us our list of \texttt{valid\_numbers}. Next, we found the incorrect
number in Blake's list. We compared \texttt{blake\_numbers} to 
\texttt{valid\_numbers} using \texttt{setdiff()}. This found which number
was in Blake's list but not in the valid numbers, meaning that this was the
incorrect number. Finally, we found the missing number that should be in the
list. This was similar to finding the incorrect number, but now we used
\texttt{setdiff(valid\_numbers, blake\_numbers)} to determine which number was in
the valid list and not Blake's, meaning this was the missing number. Through 
this analysis, we found that the incorrect number was 75 and the missing number
was 74. This makes sense, because the prime factorization of 75 is 
\(3 \times 5 \times 5\), which is not the product of \textit{distinct} prime 
numbers. On the other hand, the prime factorization of 74 is \(2 \times 37\),
which is a product of distinct prime numbers and fits in the list between 69 
and 77.

\bibliography{bibliography}
\end{document}

